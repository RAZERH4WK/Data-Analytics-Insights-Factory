% Options for packages loaded elsewhere
\PassOptionsToPackage{unicode}{hyperref}
\PassOptionsToPackage{hyphens}{url}
%
\documentclass[
]{article}
\usepackage{amsmath,amssymb}
\usepackage{lmodern}
\usepackage{ifxetex,ifluatex}
\ifnum 0\ifxetex 1\fi\ifluatex 1\fi=0 % if pdftex
  \usepackage[T1]{fontenc}
  \usepackage[utf8]{inputenc}
  \usepackage{textcomp} % provide euro and other symbols
\else % if luatex or xetex
  \usepackage{unicode-math}
  \defaultfontfeatures{Scale=MatchLowercase}
  \defaultfontfeatures[\rmfamily]{Ligatures=TeX,Scale=1}
\fi
% Use upquote if available, for straight quotes in verbatim environments
\IfFileExists{upquote.sty}{\usepackage{upquote}}{}
\IfFileExists{microtype.sty}{% use microtype if available
  \usepackage[]{microtype}
  \UseMicrotypeSet[protrusion]{basicmath} % disable protrusion for tt fonts
}{}
\makeatletter
\@ifundefined{KOMAClassName}{% if non-KOMA class
  \IfFileExists{parskip.sty}{%
    \usepackage{parskip}
  }{% else
    \setlength{\parindent}{0pt}
    \setlength{\parskip}{6pt plus 2pt minus 1pt}}
}{% if KOMA class
  \KOMAoptions{parskip=half}}
\makeatother
\usepackage{xcolor}
\IfFileExists{xurl.sty}{\usepackage{xurl}}{} % add URL line breaks if available
\IfFileExists{bookmark.sty}{\usepackage{bookmark}}{\usepackage{hyperref}}
\hypersetup{
  pdftitle={Laboratorio 1, Parte 1},
  pdfauthor={Roberto Aguilar},
  hidelinks,
  pdfcreator={LaTeX via pandoc}}
\urlstyle{same} % disable monospaced font for URLs
\usepackage[margin=1in]{geometry}
\usepackage{color}
\usepackage{fancyvrb}
\newcommand{\VerbBar}{|}
\newcommand{\VERB}{\Verb[commandchars=\\\{\}]}
\DefineVerbatimEnvironment{Highlighting}{Verbatim}{commandchars=\\\{\}}
% Add ',fontsize=\small' for more characters per line
\usepackage{framed}
\definecolor{shadecolor}{RGB}{248,248,248}
\newenvironment{Shaded}{\begin{snugshade}}{\end{snugshade}}
\newcommand{\AlertTok}[1]{\textcolor[rgb]{0.94,0.16,0.16}{#1}}
\newcommand{\AnnotationTok}[1]{\textcolor[rgb]{0.56,0.35,0.01}{\textbf{\textit{#1}}}}
\newcommand{\AttributeTok}[1]{\textcolor[rgb]{0.77,0.63,0.00}{#1}}
\newcommand{\BaseNTok}[1]{\textcolor[rgb]{0.00,0.00,0.81}{#1}}
\newcommand{\BuiltInTok}[1]{#1}
\newcommand{\CharTok}[1]{\textcolor[rgb]{0.31,0.60,0.02}{#1}}
\newcommand{\CommentTok}[1]{\textcolor[rgb]{0.56,0.35,0.01}{\textit{#1}}}
\newcommand{\CommentVarTok}[1]{\textcolor[rgb]{0.56,0.35,0.01}{\textbf{\textit{#1}}}}
\newcommand{\ConstantTok}[1]{\textcolor[rgb]{0.00,0.00,0.00}{#1}}
\newcommand{\ControlFlowTok}[1]{\textcolor[rgb]{0.13,0.29,0.53}{\textbf{#1}}}
\newcommand{\DataTypeTok}[1]{\textcolor[rgb]{0.13,0.29,0.53}{#1}}
\newcommand{\DecValTok}[1]{\textcolor[rgb]{0.00,0.00,0.81}{#1}}
\newcommand{\DocumentationTok}[1]{\textcolor[rgb]{0.56,0.35,0.01}{\textbf{\textit{#1}}}}
\newcommand{\ErrorTok}[1]{\textcolor[rgb]{0.64,0.00,0.00}{\textbf{#1}}}
\newcommand{\ExtensionTok}[1]{#1}
\newcommand{\FloatTok}[1]{\textcolor[rgb]{0.00,0.00,0.81}{#1}}
\newcommand{\FunctionTok}[1]{\textcolor[rgb]{0.00,0.00,0.00}{#1}}
\newcommand{\ImportTok}[1]{#1}
\newcommand{\InformationTok}[1]{\textcolor[rgb]{0.56,0.35,0.01}{\textbf{\textit{#1}}}}
\newcommand{\KeywordTok}[1]{\textcolor[rgb]{0.13,0.29,0.53}{\textbf{#1}}}
\newcommand{\NormalTok}[1]{#1}
\newcommand{\OperatorTok}[1]{\textcolor[rgb]{0.81,0.36,0.00}{\textbf{#1}}}
\newcommand{\OtherTok}[1]{\textcolor[rgb]{0.56,0.35,0.01}{#1}}
\newcommand{\PreprocessorTok}[1]{\textcolor[rgb]{0.56,0.35,0.01}{\textit{#1}}}
\newcommand{\RegionMarkerTok}[1]{#1}
\newcommand{\SpecialCharTok}[1]{\textcolor[rgb]{0.00,0.00,0.00}{#1}}
\newcommand{\SpecialStringTok}[1]{\textcolor[rgb]{0.31,0.60,0.02}{#1}}
\newcommand{\StringTok}[1]{\textcolor[rgb]{0.31,0.60,0.02}{#1}}
\newcommand{\VariableTok}[1]{\textcolor[rgb]{0.00,0.00,0.00}{#1}}
\newcommand{\VerbatimStringTok}[1]{\textcolor[rgb]{0.31,0.60,0.02}{#1}}
\newcommand{\WarningTok}[1]{\textcolor[rgb]{0.56,0.35,0.01}{\textbf{\textit{#1}}}}
\usepackage{graphicx}
\makeatletter
\def\maxwidth{\ifdim\Gin@nat@width>\linewidth\linewidth\else\Gin@nat@width\fi}
\def\maxheight{\ifdim\Gin@nat@height>\textheight\textheight\else\Gin@nat@height\fi}
\makeatother
% Scale images if necessary, so that they will not overflow the page
% margins by default, and it is still possible to overwrite the defaults
% using explicit options in \includegraphics[width, height, ...]{}
\setkeys{Gin}{width=\maxwidth,height=\maxheight,keepaspectratio}
% Set default figure placement to htbp
\makeatletter
\def\fps@figure{htbp}
\makeatother
\setlength{\emergencystretch}{3em} % prevent overfull lines
\providecommand{\tightlist}{%
  \setlength{\itemsep}{0pt}\setlength{\parskip}{0pt}}
\setcounter{secnumdepth}{-\maxdimen} % remove section numbering
\ifluatex
  \usepackage{selnolig}  % disable illegal ligatures
\fi

\title{Laboratorio 1, Parte 1}
\author{Roberto Aguilar}
\date{5/1/2021}

\begin{document}
\maketitle

\hypertarget{paso-3}{%
\section{Paso 3}\label{paso-3}}

Primero accedemos a los discursos en formato .txt, que se encuentran
almacenados en el directorio

\begin{Shaded}
\begin{Highlighting}[]
\CommentTok{\# Direccion de documentos en directorio}

\NormalTok{stevepath }\OtherTok{\textless{}{-}} \StringTok{\textquotesingle{}SteveJobsStanford.txt\textquotesingle{}}
\NormalTok{capath }\OtherTok{\textless{}{-}} \StringTok{\textquotesingle{}CATraspasoPresidencial.txt\textquotesingle{}}
\NormalTok{lgpath }\OtherTok{\textless{}{-}} \StringTok{\textquotesingle{}LGSTraspasoPresidencial.txt\textquotesingle{}}

\CommentTok{\# Lectura de documentos en formato .txt}

\NormalTok{text\_steve }\OtherTok{\textless{}{-}} \FunctionTok{readLines}\NormalTok{(stevepath)}
\NormalTok{text\_ca }\OtherTok{\textless{}{-}} \FunctionTok{readLines}\NormalTok{(capath)}
\NormalTok{text\_lg }\OtherTok{\textless{}{-}} \FunctionTok{readLines}\NormalTok{(lgpath)}
\end{Highlighting}
\end{Shaded}

Ahora creamos una lista de documentos, donde cada doumento tendrá la
estructura de un párrafo de discurso mediante la función
\textbf{Corpus()}, de text mining

\begin{Shaded}
\begin{Highlighting}[]
\CommentTok{\# listado de documentos en discurso de Steve Jobs y ambos presidentes de Costa Rica}

\NormalTok{docs\_st }\OtherTok{\textless{}{-}} \FunctionTok{Corpus}\NormalTok{(}\FunctionTok{VectorSource}\NormalTok{(text\_steve))}

\NormalTok{docs\_ca }\OtherTok{\textless{}{-}} \FunctionTok{Corpus}\NormalTok{(}\FunctionTok{VectorSource}\NormalTok{(text\_ca))}

\NormalTok{docs\_lg }\OtherTok{\textless{}{-}} \FunctionTok{Corpus}\NormalTok{(}\FunctionTok{VectorSource}\NormalTok{(text\_lg))}
\end{Highlighting}
\end{Shaded}

\hypertarget{paso-4}{%
\section{Paso 4}\label{paso-4}}

Ahora, nuestro interés está en insertar los datos de la forma más limpia
posible. Como vimos anteriormente en el discurso de Steve Jobs, ciertos
caracteres especiales fueron sustituídos por un ``?'' pero ahora previo
al análisis necesitaremos transformarlos en espacios en blanco para
hacer más eficiente el proceso. Para ello usaremos la función tm\_map de
text mining.

\begin{Shaded}
\begin{Highlighting}[]
\CommentTok{\# Se crea un tranformador de contenido que sustituye el caracter especial x que insertemos por un string de un espacio en blanco}

\NormalTok{toSpace }\OtherTok{\textless{}{-}} \FunctionTok{content\_transformer}\NormalTok{(}\ControlFlowTok{function}\NormalTok{ (x , pattern ) }\FunctionTok{gsub}\NormalTok{(pattern, }\StringTok{" "}\NormalTok{, x))}

\CommentTok{\# Sustituye por cada vocal}

\NormalTok{toa }\OtherTok{\textless{}{-}} \FunctionTok{content\_transformer}\NormalTok{(}\ControlFlowTok{function}\NormalTok{ (x , pattern ) }\FunctionTok{gsub}\NormalTok{(pattern, }\StringTok{"a"}\NormalTok{, x)) }
\NormalTok{toe }\OtherTok{\textless{}{-}} \FunctionTok{content\_transformer}\NormalTok{(}\ControlFlowTok{function}\NormalTok{ (x , pattern ) }\FunctionTok{gsub}\NormalTok{(pattern, }\StringTok{"e"}\NormalTok{, x)) }
\NormalTok{toi }\OtherTok{\textless{}{-}} \FunctionTok{content\_transformer}\NormalTok{(}\ControlFlowTok{function}\NormalTok{ (x , pattern ) }\FunctionTok{gsub}\NormalTok{(pattern, }\StringTok{"i"}\NormalTok{, x))}
\NormalTok{too }\OtherTok{\textless{}{-}} \FunctionTok{content\_transformer}\NormalTok{(}\ControlFlowTok{function}\NormalTok{ (x , pattern ) }\FunctionTok{gsub}\NormalTok{(pattern, }\StringTok{"o"}\NormalTok{, x)) }
\NormalTok{tou }\OtherTok{\textless{}{-}} \FunctionTok{content\_transformer}\NormalTok{(}\ControlFlowTok{function}\NormalTok{ (x , pattern ) }\FunctionTok{gsub}\NormalTok{(pattern, }\StringTok{"u"}\NormalTok{, x)) }

\CommentTok{\# Sustituye por letra ñ}

\NormalTok{ton }\OtherTok{\textless{}{-}} \FunctionTok{content\_transformer}\NormalTok{(}\ControlFlowTok{function}\NormalTok{ (x , pattern ) }\FunctionTok{gsub}\NormalTok{(pattern, }\StringTok{"ñ"}\NormalTok{, x)) }

\CommentTok{\# Aplicamos la transfomación para cada listado de documentos}

\CommentTok{\# Steve Jobs}
\NormalTok{docs\_st }\OtherTok{\textless{}{-}} \FunctionTok{tm\_map}\NormalTok{(docs\_st, toSpace, }\StringTok{"/"}\NormalTok{)}
\NormalTok{docs\_st }\OtherTok{\textless{}{-}} \FunctionTok{tm\_map}\NormalTok{(docs\_st, toSpace, }\StringTok{"@"}\NormalTok{)}
\NormalTok{docs\_st }\OtherTok{\textless{}{-}} \FunctionTok{tm\_map}\NormalTok{(docs\_st, toSpace, }\StringTok{"}\SpecialCharTok{\textbackslash{}\textbackslash{}}\StringTok{|"}\NormalTok{)}
\NormalTok{docs\_st }\OtherTok{\textless{}{-}} \FunctionTok{tm\_map}\NormalTok{(docs\_st, toSpace, }\StringTok{"ۼ"}\NormalTok{)}
\NormalTok{docs\_st }\OtherTok{\textless{}{-}} \FunctionTok{tm\_map}\NormalTok{(docs\_st, toSpace, }\StringTok{"ÿ"}\NormalTok{)}

\CommentTok{\# Luis Guillermo Solis}
\NormalTok{docs\_lg }\OtherTok{\textless{}{-}} \FunctionTok{tm\_map}\NormalTok{(docs\_lg, toSpace, }\StringTok{"/"}\NormalTok{)}
\NormalTok{docs\_lg }\OtherTok{\textless{}{-}} \FunctionTok{tm\_map}\NormalTok{(docs\_lg, toSpace, }\StringTok{"@"}\NormalTok{)}
\NormalTok{docs\_lg }\OtherTok{\textless{}{-}} \FunctionTok{tm\_map}\NormalTok{(docs\_lg, toSpace, }\StringTok{"}\SpecialCharTok{\textbackslash{}\textbackslash{}}\StringTok{|"}\NormalTok{)}
\NormalTok{docs\_lg }\OtherTok{\textless{}{-}} \FunctionTok{tm\_map}\NormalTok{(docs\_lg, toa, }\FunctionTok{c}\NormalTok{(}\StringTok{"š"}\NormalTok{,}\StringTok{" š"}\NormalTok{))}
\NormalTok{docs\_lg }\OtherTok{\textless{}{-}} \FunctionTok{tm\_map}\NormalTok{(docs\_lg, toe, }\FunctionTok{c}\NormalTok{(}\StringTok{"€"}\NormalTok{, }\StringTok{"€š"}\NormalTok{))}
\NormalTok{docs\_lg }\OtherTok{\textless{}{-}} \FunctionTok{tm\_map}\NormalTok{(docs\_lg, toi, }\FunctionTok{c}\NormalTok{(}\StringTok{"¡"}\NormalTok{, }\StringTok{"¡š"}\NormalTok{))}
\NormalTok{docs\_lg }\OtherTok{\textless{}{-}} \FunctionTok{tm\_map}\NormalTok{(docs\_lg, too, }\FunctionTok{c}\NormalTok{(}\StringTok{"¢"}\NormalTok{,}\StringTok{"¢š"}\NormalTok{))}
\NormalTok{docs\_lg }\OtherTok{\textless{}{-}} \FunctionTok{tm\_map}\NormalTok{(docs\_lg, tou, }\FunctionTok{c}\NormalTok{(}\StringTok{"£"}\NormalTok{,}\StringTok{"£š"}\NormalTok{))}
\NormalTok{docs\_lg }\OtherTok{\textless{}{-}} \FunctionTok{tm\_map}\NormalTok{(docs\_lg, ton, }\StringTok{"¤"}\NormalTok{) }
\NormalTok{docs\_lg }\OtherTok{\textless{}{-}} \FunctionTok{tm\_map}\NormalTok{(docs\_lg, toe, }\StringTok{"‚"}\NormalTok{) }

\CommentTok{\# Carlos Alvarado}
\NormalTok{docs\_ca }\OtherTok{\textless{}{-}} \FunctionTok{tm\_map}\NormalTok{(docs\_ca, toSpace, }\StringTok{"/"}\NormalTok{)}
\NormalTok{docs\_ca }\OtherTok{\textless{}{-}} \FunctionTok{tm\_map}\NormalTok{(docs\_ca, toSpace, }\StringTok{"@"}\NormalTok{)}
\NormalTok{docs\_ca }\OtherTok{\textless{}{-}} \FunctionTok{tm\_map}\NormalTok{(docs\_ca, toSpace, }\StringTok{"}\SpecialCharTok{\textbackslash{}\textbackslash{}}\StringTok{|"}\NormalTok{)}
\NormalTok{docs\_ca }\OtherTok{\textless{}{-}} \FunctionTok{tm\_map}\NormalTok{(docs\_ca, toSpace, }\StringTok{"“"}\NormalTok{)}
\NormalTok{docs\_ca }\OtherTok{\textless{}{-}} \FunctionTok{tm\_map}\NormalTok{(docs\_ca, toSpace, }\StringTok{"¡"}\NormalTok{)}
\end{Highlighting}
\end{Shaded}

Primero hacemos ajustes al discurso de Steve Jobs

\begin{Shaded}
\begin{Highlighting}[]
\CommentTok{\# Convertir el texto en lower case}
\NormalTok{docs\_st }\OtherTok{\textless{}{-}} \FunctionTok{tm\_map}\NormalTok{(docs\_st, }\FunctionTok{content\_transformer}\NormalTok{(tolower))}
\CommentTok{\# Eliminar numeros}
\NormalTok{docs\_st }\OtherTok{\textless{}{-}} \FunctionTok{tm\_map}\NormalTok{(docs\_st, removeNumbers)}
\CommentTok{\# Eliminar stopwords en ingles}
\NormalTok{docs\_st }\OtherTok{\textless{}{-}} \FunctionTok{tm\_map}\NormalTok{(docs\_st, removeWords, }\FunctionTok{stopwords}\NormalTok{(}\StringTok{"english"}\NormalTok{))}
\CommentTok{\# Eliminar stopwords personalizados}
\NormalTok{docs\_st }\OtherTok{\textless{}{-}} \FunctionTok{tm\_map}\NormalTok{(docs\_st, removeWords, }\FunctionTok{c}\NormalTok{(}\StringTok{"m"}\NormalTok{, }\StringTok{"ll"}\NormalTok{, }\StringTok{"ve"}\NormalTok{, }\StringTok{"s"}\NormalTok{))}
\CommentTok{\# Eliminar puntuaciones}
\NormalTok{docs\_st }\OtherTok{\textless{}{-}} \FunctionTok{tm\_map}\NormalTok{(docs\_st, removePunctuation)}
\CommentTok{\# Elimnar espacios extra en blanco}
\NormalTok{docs\_st }\OtherTok{\textless{}{-}} \FunctionTok{tm\_map}\NormalTok{(docs\_st, stripWhitespace)}
\CommentTok{\# Eliminamos el caracter especial}
\NormalTok{docs\_st }\OtherTok{\textless{}{-}} \FunctionTok{tm\_map}\NormalTok{(docs\_st, toSpace, }\StringTok{""}\NormalTok{)}
\end{Highlighting}
\end{Shaded}

Lo hacemos para el discurso de Luis Guillermo Solis

\begin{Shaded}
\begin{Highlighting}[]
\CommentTok{\# Convertir el texto en lower case}
\NormalTok{docs\_lg }\OtherTok{\textless{}{-}} \FunctionTok{tm\_map}\NormalTok{(docs\_lg, }\FunctionTok{content\_transformer}\NormalTok{(tolower))}
\CommentTok{\# Eliminar numeros}
\NormalTok{docs\_lg }\OtherTok{\textless{}{-}} \FunctionTok{tm\_map}\NormalTok{(docs\_lg, removeNumbers)}
\CommentTok{\# Eliminar stopwords en ingles}
\NormalTok{docs\_lg }\OtherTok{\textless{}{-}} \FunctionTok{tm\_map}\NormalTok{(docs\_lg, removeWords, }\FunctionTok{stopwords}\NormalTok{(}\StringTok{"spanish"}\NormalTok{))}
\CommentTok{\# Eliminar puntuaciones}
\NormalTok{docs\_lg }\OtherTok{\textless{}{-}} \FunctionTok{tm\_map}\NormalTok{(docs\_lg, removePunctuation)}
\CommentTok{\# Elimnar espacios extra en blanco}
\NormalTok{docs\_lg }\OtherTok{\textless{}{-}} \FunctionTok{tm\_map}\NormalTok{(docs\_lg, stripWhitespace)}
\CommentTok{\# Ajustamos la "a + espacio" }
\NormalTok{docs\_lg }\OtherTok{\textless{}{-}} \FunctionTok{tm\_map}\NormalTok{(docs\_lg, toa, }\StringTok{"â "}\NormalTok{)}
\CommentTok{\# Eliminamos el caracter especial}
\NormalTok{docs\_lg }\OtherTok{\textless{}{-}} \FunctionTok{tm\_map}\NormalTok{(docs\_lg, toSpace, }\StringTok{""}\NormalTok{)}
\CommentTok{\# Eliminar stopwords personalizados}
\NormalTok{docs\_lg }\OtherTok{\textless{}{-}} \FunctionTok{tm\_map}\NormalTok{(docs\_lg, removeWords, }\FunctionTok{c}\NormalTok{(}\StringTok{"si"}\NormalTok{, }\StringTok{"ser"}\NormalTok{, }\StringTok{"sino"}\NormalTok{, }\StringTok{"cada"}\NormalTok{, }\StringTok{"tras"}\NormalTok{, }\StringTok{"la"}\NormalTok{, }\StringTok{"el"}\NormalTok{, }\StringTok{"las"}\NormalTok{ , }\StringTok{"los"}\NormalTok{, }\StringTok{"mas"}\NormalTok{, }\StringTok{"m�s"}\NormalTok{ ,                                       }\StringTok{"que"}\NormalTok{, }\StringTok{"con"}\NormalTok{, }\StringTok{"por"}\NormalTok{, }\StringTok{"del"}\NormalTok{, }\StringTok{"una"}\NormalTok{, }\StringTok{"como"}\NormalTok{, }\StringTok{"para"}\NormalTok{, }\StringTok{"sus"}\NormalTok{, }\StringTok{"cuando"}\NormalTok{, }\StringTok{"ese"}\NormalTok{, }\StringTok{"esa"}\NormalTok{,}
                                          \StringTok{"esta"}\NormalTok{, }\StringTok{"que,"}\NormalTok{, }\StringTok{"pero"}\NormalTok{, }\StringTok{"sin"}\NormalTok{, }\StringTok{"sino"}\NormalTok{,}\StringTok{"mas"}\NormalTok{))}
\end{Highlighting}
\end{Shaded}

Y también lo hacemos para el discurso de Carlos Alvarado

\begin{Shaded}
\begin{Highlighting}[]
\CommentTok{\# Convertir el texto en lower case}
\NormalTok{docs\_ca }\OtherTok{\textless{}{-}} \FunctionTok{tm\_map}\NormalTok{(docs\_ca, }\FunctionTok{content\_transformer}\NormalTok{(tolower))}
\CommentTok{\# Eliminar numeros}
\NormalTok{docs\_ca }\OtherTok{\textless{}{-}} \FunctionTok{tm\_map}\NormalTok{(docs\_ca, removeNumbers)}
\CommentTok{\# Eliminar stopwords en ingles}
\NormalTok{docs\_ca }\OtherTok{\textless{}{-}} \FunctionTok{tm\_map}\NormalTok{(docs\_ca, removeWords, }\FunctionTok{stopwords}\NormalTok{(}\StringTok{"spanish"}\NormalTok{))}
\CommentTok{\# Eliminar stopwords personalizados}
\NormalTok{docs\_ca }\OtherTok{\textless{}{-}} \FunctionTok{tm\_map}\NormalTok{(docs\_ca, removeWords, }\FunctionTok{c}\NormalTok{(}\StringTok{"si"}\NormalTok{, }\StringTok{"ser"}\NormalTok{, }\StringTok{"sino"}\NormalTok{, }\StringTok{"cada"}\NormalTok{, }\StringTok{"tras"}\NormalTok{, }\StringTok{"la"}\NormalTok{, }\StringTok{"el"}\NormalTok{, }\StringTok{"las"}\NormalTok{ , }\StringTok{"los"}\NormalTok{, }\StringTok{"mas"}\NormalTok{, }\StringTok{"m�s"}\NormalTok{ ,                                       }\StringTok{"que"}\NormalTok{, }\StringTok{"con"}\NormalTok{, }\StringTok{"por"}\NormalTok{, }\StringTok{"del"}\NormalTok{, }\StringTok{"una"}\NormalTok{, }\StringTok{"como"}\NormalTok{, }\StringTok{"para"}\NormalTok{, }\StringTok{"sus"}\NormalTok{, }\StringTok{"cuando"}\NormalTok{, }\StringTok{"ese"}\NormalTok{, }\StringTok{"esa"}\NormalTok{,}
                                          \StringTok{"esta"}\NormalTok{, }\StringTok{"que,"}\NormalTok{, }\StringTok{"pero"}\NormalTok{, }\StringTok{"sin"}\NormalTok{, }\StringTok{"sino"}\NormalTok{))}
\CommentTok{\# Eliminar puntuaciones}
\NormalTok{docs\_ca }\OtherTok{\textless{}{-}} \FunctionTok{tm\_map}\NormalTok{(docs\_ca, removePunctuation)}
\CommentTok{\# Elimnar espacios extra en blanco}
\NormalTok{docs\_ca }\OtherTok{\textless{}{-}} \FunctionTok{tm\_map}\NormalTok{(docs\_ca, stripWhitespace)}
\CommentTok{\# Ajustamos la "a + espacio" }
\NormalTok{docs\_ca }\OtherTok{\textless{}{-}} \FunctionTok{tm\_map}\NormalTok{(docs\_ca, toa, }\StringTok{"â "}\NormalTok{)}
\CommentTok{\# Eliminamos el caracter especial}
\NormalTok{docs\_ca }\OtherTok{\textless{}{-}} \FunctionTok{tm\_map}\NormalTok{(docs\_ca, toSpace, }\StringTok{""}\NormalTok{)}
\end{Highlighting}
\end{Shaded}

\hypertarget{paso-6}{%
\section{Paso 6}\label{paso-6}}

Ahora creamos una tabla matriz con la frecuencia de aparacion de
palabras para cada discurso

\begin{Shaded}
\begin{Highlighting}[]
\CommentTok{\# Steve Jobs}
\NormalTok{dtmst }\OtherTok{\textless{}{-}} \FunctionTok{TermDocumentMatrix}\NormalTok{(docs\_st)}
\NormalTok{mst }\OtherTok{\textless{}{-}} \FunctionTok{as.matrix}\NormalTok{(dtmst)}
\NormalTok{vst }\OtherTok{\textless{}{-}} \FunctionTok{sort}\NormalTok{(}\FunctionTok{rowSums}\NormalTok{(mst),}\AttributeTok{decreasing=}\ConstantTok{TRUE}\NormalTok{)}
\NormalTok{dst }\OtherTok{\textless{}{-}} \FunctionTok{data.frame}\NormalTok{(}\AttributeTok{word =} \FunctionTok{names}\NormalTok{(vst),}\AttributeTok{freq=}\NormalTok{vst)}
\FunctionTok{head}\NormalTok{(dst, }\DecValTok{10}\NormalTok{)}
\end{Highlighting}
\end{Shaded}

\begin{verbatim}
##            word freq
## life       life   17
## college college   14
## one         one    9
## years     years    9
## apple     apple    9
## just       just    8
## never     never    8
## dropped dropped    7
## months   months    7
## later     later    7
\end{verbatim}

\begin{Shaded}
\begin{Highlighting}[]
\CommentTok{\# Luis Guillermo}
\NormalTok{dtmlg }\OtherTok{\textless{}{-}} \FunctionTok{TermDocumentMatrix}\NormalTok{(docs\_lg)}
\NormalTok{mlg }\OtherTok{\textless{}{-}} \FunctionTok{as.matrix}\NormalTok{(dtmlg)}
\NormalTok{vlg }\OtherTok{\textless{}{-}} \FunctionTok{sort}\NormalTok{(}\FunctionTok{rowSums}\NormalTok{(mlg),}\AttributeTok{decreasing=}\ConstantTok{TRUE}\NormalTok{)}
\NormalTok{dlg }\OtherTok{\textless{}{-}} \FunctionTok{data.frame}\NormalTok{(}\AttributeTok{word =} \FunctionTok{names}\NormalTok{(vlg),}\AttributeTok{freq=}\NormalTok{vlg)}
\FunctionTok{head}\NormalTok{(dlg, }\DecValTok{10}\NormalTok{)}
\end{Highlighting}
\end{Shaded}

\begin{verbatim}
##                  word freq
## democracia democracia   13
## pueblo         pueblo   13
## pais             pais   12
## rica             rica   11
## costa           costa   10
## nacional     nacional    9
## tambien       tambien    9
## gobierno     gobierno    8
## sociedad     sociedad    8
## solo             solo    8
\end{verbatim}

\begin{Shaded}
\begin{Highlighting}[]
\CommentTok{\# Carlos Alvarado}
\NormalTok{dtmca }\OtherTok{\textless{}{-}} \FunctionTok{TermDocumentMatrix}\NormalTok{(docs\_ca)}
\NormalTok{mca }\OtherTok{\textless{}{-}} \FunctionTok{as.matrix}\NormalTok{(dtmca)}
\NormalTok{vca }\OtherTok{\textless{}{-}} \FunctionTok{sort}\NormalTok{(}\FunctionTok{rowSums}\NormalTok{(mca),}\AttributeTok{decreasing=}\ConstantTok{TRUE}\NormalTok{)}
\NormalTok{dca }\OtherTok{\textless{}{-}} \FunctionTok{data.frame}\NormalTok{(}\AttributeTok{word =} \FunctionTok{names}\NormalTok{(vca),}\AttributeTok{freq=}\NormalTok{vca)}
\FunctionTok{head}\NormalTok{(dca, }\DecValTok{10}\NormalTok{)}
\end{Highlighting}
\end{Shaded}

\begin{verbatim}
##                          word freq
## pais                     pais   22
## personas             personas   19
## debemos               debemos   17
## bienestar           bienestar   14
## bicentenario     bicentenario   14
## costa                   costa   14
## rica                     rica   14
## educacion           educacion   13
## todas                   todas   11
## costarricenses costarricenses   10
\end{verbatim}

\hypertarget{paso-7}{%
\section{Paso 7}\label{paso-7}}

Procedemos a ilustrar con un Wordcloud para cada discurso

\hypertarget{discurso-de-steve-jobs}{%
\subsubsection{Discurso de Steve Jobs}\label{discurso-de-steve-jobs}}

\begin{Shaded}
\begin{Highlighting}[]
\FunctionTok{set.seed}\NormalTok{(}\DecValTok{1234}\NormalTok{)}
\FunctionTok{wordcloud}\NormalTok{(}\AttributeTok{words =}\NormalTok{ dst}\SpecialCharTok{$}\NormalTok{word, }\AttributeTok{freq =}\NormalTok{ dst}\SpecialCharTok{$}\NormalTok{freq, }\AttributeTok{min.freq =} \DecValTok{1}\NormalTok{, }\AttributeTok{max.words=}\DecValTok{100}\NormalTok{,}
          \AttributeTok{random.order=}\ConstantTok{FALSE}\NormalTok{, }\AttributeTok{rot.per=}\FloatTok{0.35}\NormalTok{, }\AttributeTok{colors=}\FunctionTok{brewer.pal}\NormalTok{(}\DecValTok{8}\NormalTok{, }\StringTok{"Dark2"}\NormalTok{))}
\end{Highlighting}
\end{Shaded}

\includegraphics{Laboratorio-1_files/figure-latex/unnamed-chunk-8-1.pdf}

\hypertarget{discurso-de-luis-guillermo-solis}{%
\subsubsection{Discurso de Luis Guillermo
Solis}\label{discurso-de-luis-guillermo-solis}}

\begin{Shaded}
\begin{Highlighting}[]
\FunctionTok{set.seed}\NormalTok{(}\DecValTok{1234}\NormalTok{)}
\FunctionTok{wordcloud}\NormalTok{(}\AttributeTok{words =}\NormalTok{ dlg}\SpecialCharTok{$}\NormalTok{word, }\AttributeTok{freq =}\NormalTok{ dlg}\SpecialCharTok{$}\NormalTok{freq, }\AttributeTok{min.freq =} \DecValTok{1}\NormalTok{, }\AttributeTok{max.words=}\DecValTok{100}\NormalTok{,}
          \AttributeTok{random.order=}\ConstantTok{FALSE}\NormalTok{, }\AttributeTok{rot.per=}\FloatTok{0.35}\NormalTok{, }\AttributeTok{colors=}\FunctionTok{brewer.pal}\NormalTok{(}\DecValTok{8}\NormalTok{, }\StringTok{"Dark2"}\NormalTok{))}
\end{Highlighting}
\end{Shaded}

\includegraphics{Laboratorio-1_files/figure-latex/unnamed-chunk-9-1.pdf}

\hypertarget{discurso-de-carlos-alvarado}{%
\subsubsection{Discurso de Carlos
Alvarado}\label{discurso-de-carlos-alvarado}}

\begin{Shaded}
\begin{Highlighting}[]
\FunctionTok{set.seed}\NormalTok{(}\DecValTok{1234}\NormalTok{)}
\FunctionTok{wordcloud}\NormalTok{(}\AttributeTok{words =}\NormalTok{ dca}\SpecialCharTok{$}\NormalTok{word, }\AttributeTok{freq =}\NormalTok{ dca}\SpecialCharTok{$}\NormalTok{freq, }\AttributeTok{min.freq =} \DecValTok{1}\NormalTok{, }\AttributeTok{max.words=}\DecValTok{100}\NormalTok{,}
          \AttributeTok{random.order=}\ConstantTok{FALSE}\NormalTok{, }\AttributeTok{rot.per=}\FloatTok{0.35}\NormalTok{, }\AttributeTok{colors=}\FunctionTok{brewer.pal}\NormalTok{(}\DecValTok{8}\NormalTok{, }\StringTok{"Dark2"}\NormalTok{))}
\end{Highlighting}
\end{Shaded}

\includegraphics{Laboratorio-1_files/figure-latex/unnamed-chunk-10-1.pdf}

\hypertarget{paso-8}{%
\section{Paso 8}\label{paso-8}}

Como análisis adicional, buscamos aquellas palabras que cómo mínimo
aparezcan 4 veces dentros de cada discurso, la asociación de los
términos

\hypertarget{discurso-de-steve-jobs-1}{%
\subsubsection{Discurso de Steve Jobs}\label{discurso-de-steve-jobs-1}}

\begin{Shaded}
\begin{Highlighting}[]
\CommentTok{\# Frecuencia}
\FunctionTok{findFreqTerms}\NormalTok{(dtmst, }\AttributeTok{lowfreq =} \DecValTok{4}\NormalTok{)}
\end{Highlighting}
\end{Shaded}

\begin{verbatim}
##  [1] "college"    "ever"       "just"       "life"       "never"     
##  [6] "one"        "today"      "want"       "dots"       "first"     
## [11] "story"      "dropped"    "months"     "course"     "decided"   
## [16] "everything" "found"      "got"        "later"      "parents"   
## [21] "put"        "started"    "best"       "didnt"      "going"     
## [26] "looking"    "made"       "work"       "years"      "every"     
## [31] "get"        "let"        "loved"      "turned"     "great"     
## [36] "way"        "computer"   "connect"    "even"       "something" 
## [41] "love"       "apple"      "company"    "fired"      "know"      
## [46] "heart"      "next"       "now"        "dont"       "find"      
## [51] "death"      "die"        "stay"
\end{verbatim}

\begin{Shaded}
\begin{Highlighting}[]
\CommentTok{\# Asociación con "college"}
\FunctionTok{findAssocs}\NormalTok{(dtmst, }\AttributeTok{terms =} \StringTok{"college"}\NormalTok{, }\AttributeTok{corlimit =} \FloatTok{0.3}\NormalTok{)}
\end{Highlighting}
\end{Shaded}

\begin{verbatim}
## $college
##       minute       wanted      parents      decided        later    graduated 
##         0.87         0.87         0.80         0.72         0.69         0.66 
##      adopted     adoption       asking         baby   biological        birth 
##         0.60         0.60         0.60         0.60         0.60         0.60 
##         born          boy         call       except       father         girl 
##         0.60         0.60         0.60         0.60         0.60         0.60 
##    graduates         high       lawyer         list       middle       mother 
##         0.60         0.60         0.60         0.60         0.60         0.60 
##       papers       popped     promised      refused     relented         said 
##         0.60         0.60         0.60         0.60         0.60         0.60 
##       school          set         sign     strongly      student   unexpected 
##         0.60         0.60         0.60         0.60         0.60         0.60 
##        unwed      waiting        chose      couldnt    decisions     dropping 
##         0.60         0.60         0.60         0.60         0.60         0.60 
##    expensive       figure         idea     interest  interesting        money 
##         0.60         0.60         0.60         0.60         0.60         0.60 
##      naively         ones     required        saved      savings        scary 
##         0.60         0.60         0.60         0.60         0.60         0.60 
##     spending        spent     stanford         stop       taking      tuition 
##         0.60         0.60         0.60         0.60         0.60         0.60 
##        value workingclass        never      classes         back       course 
##         0.60         0.60         0.53         0.50         0.46         0.44 
##      dropped         felt     graduate        night        young        begin 
##         0.43         0.37         0.37         0.37         0.37         0.37 
##       entire         help       looked       pretty          see          six 
##         0.37         0.37         0.37         0.37         0.37         0.37 
##       months       really         time         drop 
##         0.36         0.36         0.36         0.35
\end{verbatim}

Ahora tenemos la frecuencia de forma visual

\begin{Shaded}
\begin{Highlighting}[]
\CommentTok{\# Grafico de barras}
\FunctionTok{barplot}\NormalTok{(dst[}\DecValTok{1}\SpecialCharTok{:}\DecValTok{10}\NormalTok{,]}\SpecialCharTok{$}\NormalTok{freq, }\AttributeTok{las =} \DecValTok{2}\NormalTok{, }\AttributeTok{names.arg =}\NormalTok{ dst[}\DecValTok{1}\SpecialCharTok{:}\DecValTok{10}\NormalTok{,]}\SpecialCharTok{$}\NormalTok{word, }\AttributeTok{col =}\StringTok{"lightblue"}\NormalTok{, main}
\OtherTok{=}\StringTok{"Palabras más frecuentes de Steve Jobs"}\NormalTok{, }\AttributeTok{ylab =} \StringTok{"Frecuencia de las Palabras"}\NormalTok{)}
\end{Highlighting}
\end{Shaded}

\includegraphics{Laboratorio-1_files/figure-latex/unnamed-chunk-12-1.pdf}

\hypertarget{discurso-de-luis-guillermo-solis-1}{%
\subsubsection{Discurso de Luis Guillermo
Solis}\label{discurso-de-luis-guillermo-solis-1}}

\begin{Shaded}
\begin{Highlighting}[]
\CommentTok{\# Frecuencia}
\FunctionTok{findFreqTerms}\NormalTok{(dtmlg, }\AttributeTok{lowfreq =} \DecValTok{4}\NormalTok{)}
\end{Highlighting}
\end{Shaded}

\begin{verbatim}
##  [1] "administracion"  "bienestar"       "cambio"          "capacidad"      
##  [5] "ciudadana"       "clamor"          "constitucion"    "corrupcion"     
##  [9] "costa"           "costarricense"   "costarricenses"  "democracia"     
## [13] "desarrollo"      "dialogo"         "ello"            "esfuerzo"       
## [17] "espiritu"        "forma"           "formas"          "fuerzas"        
## [21] "gobierno"        "hace"            "hacia"           "haremos"        
## [25] "hicieron"        "himno"           "historia"        "hoy"            
## [29] "mandato"         "medio"           "mejor"           "nacional"       
## [33] "nueva"           "nuevas"          "pais"            "participacion"  
## [37] "poder"           "politica"        "politicas"       "produccion"     
## [41] "publica"         "publicas"        "publicos"        "pueblo"         
## [45] "puede"           "pues"            "republica"       "respeto"        
## [49] "responsabilidad" "rica"            "riqueza"         "siempre"        
## [53] "social"          "sociales"        "sociedad"        "solo"           
## [57] "tambien"         "todas"           "transparencia"   "voluntad"
\end{verbatim}

Ahora tenemos la frecuencia de forma visual

\begin{Shaded}
\begin{Highlighting}[]
\CommentTok{\# Grafico de barras}
\FunctionTok{barplot}\NormalTok{(dlg[}\DecValTok{1}\SpecialCharTok{:}\DecValTok{10}\NormalTok{,]}\SpecialCharTok{$}\NormalTok{freq, }\AttributeTok{las =} \DecValTok{2}\NormalTok{, }\AttributeTok{names.arg =}\NormalTok{ dlg[}\DecValTok{1}\SpecialCharTok{:}\DecValTok{10}\NormalTok{,]}\SpecialCharTok{$}\NormalTok{word, }\AttributeTok{col =}\StringTok{"lightblue"}\NormalTok{, main}
\OtherTok{=}\StringTok{"Palabras más frecuentes de Luis Guillermo Solis"}\NormalTok{, }\AttributeTok{ylab =} \StringTok{"Frecuencia de las Palabras"}\NormalTok{)}
\end{Highlighting}
\end{Shaded}

\includegraphics{Laboratorio-1_files/figure-latex/unnamed-chunk-14-1.pdf}

\hypertarget{discurso-de-carlos-alvarado-1}{%
\subsubsection{Discurso de Carlos
Alvarado}\label{discurso-de-carlos-alvarado-1}}

\begin{Shaded}
\begin{Highlighting}[]
\CommentTok{\# Frecuencia}
\FunctionTok{findFreqTerms}\NormalTok{(dtmca, }\AttributeTok{lowfreq =} \DecValTok{4}\NormalTok{)}
\end{Highlighting}
\end{Shaded}

\begin{verbatim}
##  [1] "costarricenses"  "asi"             "bienestar"       "especial"       
##  [5] "hoy"             "importantes"     "llevar"          "mayor"          
##  [9] "mejor"           "mismo"           "pais"            "personas"       
## [13] "todas"           "bicentenario"    "costa"           "lograr"         
## [17] "mano"            "paso"            "rica"            "vida"           
## [21] "mil"             "mujeres"         "primeros"        "solo"           
## [25] "anos"            "tener"           "vision"          "paz"            
## [29] "trabajo"         "construyo"       "siempre"         "ciudadania"     
## [33] "jose"            "nacional"        "tarea"           "bien"           
## [37] "dialogo"         "gracias"         "hacer"           "historia"       
## [41] "voluntad"        "creacion"        "desarrollo"      "educacion"      
## [45] "publica"         "dar"             "derechos"        "humanos"        
## [49] "mundo"           "san"             "tambien"         "generacion"     
## [53] "materia"         "fiscal"          "reforma"         "sistema"        
## [57] "seguridad"       "social"          "avanzar"         "tiempo"         
## [61] "haremos"         "sabemos"         "sera"            "trabajar"       
## [65] "podemos"         "cambio"          "frente"          "gobierno"       
## [69] "toda"            "norte"           "debemos"         "futuro"         
## [73] "actuar"          "debe"            "sociedad"        "empleo"         
## [77] "infraestructura" "salud"           "calidad"         "ruta"           
## [81] "uso"             "manera"          "transporte"      "publico"
\end{verbatim}

Ahora tenemos la frecuencia de forma visual

\begin{Shaded}
\begin{Highlighting}[]
\CommentTok{\# Grafico de barras}
\FunctionTok{barplot}\NormalTok{(dca[}\DecValTok{1}\SpecialCharTok{:}\DecValTok{10}\NormalTok{,]}\SpecialCharTok{$}\NormalTok{freq, }\AttributeTok{las =} \DecValTok{2}\NormalTok{, }\AttributeTok{names.arg =}\NormalTok{ dca[}\DecValTok{1}\SpecialCharTok{:}\DecValTok{10}\NormalTok{,]}\SpecialCharTok{$}\NormalTok{word, }\AttributeTok{col =}\StringTok{"lightblue"}\NormalTok{, main}
\OtherTok{=}\StringTok{"Palabras más frecuentes de Carlos Alvarado"}\NormalTok{, }\AttributeTok{ylab =} \StringTok{"Frecuencia de las Palabras"}\NormalTok{)}
\end{Highlighting}
\end{Shaded}

\includegraphics{Laboratorio-1_files/figure-latex/unnamed-chunk-16-1.pdf}

\end{document}
